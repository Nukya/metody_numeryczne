\subsection*{Część 2.}

W części drugiej przeprowadzono porównanie trzech metod wyznaczania funkcji $h(\Delta T)$: 
interpolacji Lagrange'a, aproksymacji metodą najmniejszych kwadratów (MNK) oraz funkcji sklejanych 
trzeciego stopnia. Zgodnie z wymaganiami zadania, dla metod interpolacyjnych wygenerowano 
równoodległe węzły w przedziale danych pomiarowych, a wartości funkcji w tych punktach określono 
za pomocą interpolacji liniowej.

\begin{figure}[H]
	\centering
	\includegraphics[height=0.33\textheight]{porownanie aproksymacji.png}
	\caption{Porównanie różnych metod aproksymacji}
\end{figure}

\begin{table}[H]
	\centering
	\renewcommand{\arraystretch}{1.3}
	\caption{Błędy bezwzględne i względne}
	
	\begin{tabular}{|l|c|c|c|c|}
		\hline
		\multirow{2}{*}{\textbf{Metoda}} &
		\multicolumn{2}{c|}{\textbf{Błąd bezwzględny}} &
		\multicolumn{2}{c|}{\textbf{Błąd względny}} \\
		\cline{2-5}
		& \textbf{Średni} & \textbf{Maksymalny} & \textbf{Średni} & \textbf{Maksymalny} \\
		\hline
		MNK      & 1.5752 & 4.7903 & 0.95511 & 2.8514 \\
		Lagrange & 1.1781 & 3.5873 & 0.71285 & 2.0382 \\
		Splajn   & 1.0242 & 2.2889 & 0.63184 & 1.4306 \\
		\hline
	\end{tabular}
\end{table}


Interpolacja Lagrange'a pozwoliła odtworzyć przebieg funkcji w równoodległych węzłach, jednak 
charakteryzowała się znacznymi oscylacjami, co jest typowym przejawem efektu Rungego. Zjawisko to 
obniża dokładność oraz stabilność metody, co czyni ją nieprzydatną w analizowanym zadaniu.

Splajn kubiczny, również oparty na równoodległych danych, wykazał znacznie lepsze własności 
aproksymacyjne. Obliczone błędy średnie i maksymalne — zarówno względne, jak i bezwzględne — 
okazały się najniższe spośród wszystkich trzech metod, co jednoznacznie wskazuje na najwyższą 
dokładność splajnu. Uzyskany przebieg jest gładki i stabilny, a jednocześnie wiernie odtwarza 
pomiarowy charakter funkcji.

Aproksymacja MNK, mimo że nie osiągnęła najniższych błędów, cechuje się dużą stabilnością 
numeryczną i odpornością na szum pomiarowy. Z tego względu jest metodą najbardziej praktyczną 
do dalszego wykorzystania w modelu chłodzenia, zwłaszcza że wymaga ona jedynie wyznaczenia 
parametrów wielomianu, a nie generowania równoodległych danych.

\begin{figure}[H]
	\centering
	\includegraphics[height=0.33\textheight]{przebieg temperatury cieczy dla roznych metod aproksymacji.png}
	\caption{Przebiegi temperatury cieczy dla różnych metod aproksymacji}
\end{figure}

\begin{figure}[H]
	\centering
	\includegraphics[height=0.33\textheight]{przebieg temperatury preta dla roznych metod aproksymacji.png}
	\caption{Przebiegi temperatury pręta dla różnych metod aproksymacji}
\end{figure}

Wykorzystanie uzyskanej funkcji $h(\Delta T)$ w nieliniowym modelu chłodzenia umożliwiło otrzymanie 
fizycznie poprawnych przebiegów temperatur. Temperatura pręta szybko maleje, natomiast temperatura 
oleju wzrasta do poziomu równowagi, po czym układ stabilizuje się. Model zachował stabilność 
numeryczną w całym analizowanym przedziale czasu, co potwierdza poprawność doboru metody MNK 
jako podstawy nieliniowego opisu wymiany ciepła.
