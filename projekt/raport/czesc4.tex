\section{część 4. projektu}

\subsection{Cel badawczy}

Celem czwartej części projektu było wyznaczenie minimalnej masy oleju chłodzącego, która pozwala schłodzić pręt do temperatury $125\degree$C w czasie $t=0{,}7$ s. W tym celu wykorzystano pełny, nieliniowy model chłodzenia wraz z aproksymowaną funkcją współczynnika wnikania ciepła $h(\Delta T)$ otrzymaną z funkcji sklejanych kubicznych. Rozwiązanie równania $T_b(m_w, 0{,}7\ \mathrm{s}) = 125\degree\mathrm{C}$ przeprowadzono metodą Newtona--Raphsona.

\subsection{Przebieg eksperymentu}

W kolejnych iteracjach Newtona wykonywano symulację chłodzenia dla zadanej masy oleju i obliczano przybliżoną pochodną temperatury pręta względem masy:
\[
T'_b(m_w) \approx \frac{T_b(m_w + \Delta m_w) - T_b(m_w)}{\Delta m_w}.
\]
Aktualizacja wartości masy przebiegała zgodnie ze wzorem:
\[
m_w^{(i+1)} = m_w^{(i)} - 
\frac{T_b(m_w^{(i)}) - 125}{T_b'(m_w^{(i)})}.
\]
Jako punkt startowy przyjęto $m_w^{(0)} = 2{,}0$ kg, a wartość różnicową $\Delta m_w = 0{,}02$ kg. Do całkowania równań różniczkowych wykorzystano ulepszoną metodę Eulera.

\subsection{Analiza wyników}

\subsubsection{Przebieg iteracji Newtona}

W Tabeli~\ref{tab:newton} przedstawiono pełny przebieg iteracji metody Newtona--Raphsona. W pierwszych krokach obserwowano niestabilność spowodowaną dużym oddaleniem punktu startowego od właściwego rozwiązania. W kolejnych iteracjach algorytm szybko zbiegał do obszaru stabilnego i osiągnął wysoką dokładność końcową.

\begin{table}[H]
	\centering
	\caption{Iteracje metody Newtona-Raphsona dla równania $T_b(m_w)=125\degree$C.}
	\label{tab:newton}
	\begin{tabular}{|c|c|c|c|c|}
		\hline
		Iteracja $i$ &
		$m_w^{(i)}$ [kg] &
		$T_b(m_w^{(i)})$ [$\degree$C] &
		$T_b'(m_w^{(i)})$ [$\degree$C/kg] &
		$\Delta T_b = T_b - T_\mathrm{cel}$ [$\degree$C] \\
		\hline
		1 & 2,00000  & 35,0992  &  -4,957036  & -89,900794 \\
		2 & 1,00000  & 45,0262  & -19,305357  & -79,973757 \\
		3 & 0,50000  & 64,3812  & -73,289308  & -60,618750 \\
		4 & 0,25000  & 101,2083 & -265,220581 & -23,791726 \\
		5 & 0,16029  & 139,6936 & -580,334298 &  14,693621 \\
		6 & 0,18561  & 125,3851 & -450,252403 &   0,385108 \\
		7 & 0,18647  & 124,9638 & -446,649355 &  -0,036175 \\
		8 & 0,18639  & 125,0036 & -446,988678 &   0,003566 \\
		9 & 0,18640  & 124,9997 & -446,955240 &  -0,000350 \\
		\hline
	\end{tabular}
\end{table}

Na podstawie uzyskanych wyników wyznaczono wartość:
\begin{displaymath}
	m_w \approx 0{,}1864\ \text{kg},
\end{displaymath}


\subsection{Wnioski końcowe}

Metoda Newtona-Raphsona okazała się skuteczną i szybką metodą rozwiązywania równania odwrotnego analizowanego w tej części projektu. Po kilku iteracjach osiągnięto wysoką dokładność rozwiązania. Jednocześnie zauważono, że model jest bardzo czuły na wartość masy oleju, co przejawia się dużą wartością pochodnej $T'_b(m_w)$ w pobliżu rozwiązania. Ponadto, uzyskany wynik leży daleko poza zakresem danych pomiarowych, co ogranicza jego interpretację fizyczną. W praktycznych zastosowaniach inżynierskich należałoby narzucić ograniczenia na minimalną masę chłodziwa tak, aby obliczenia były prowadzone wyłącznie w obszarze popartym eksperymentalnie.

